\section{Criticalities}
\label{criticalities}

The results outlined presented in the study underline the strong
evidence for a long run stable relation among macroeconomic
fundamentals and FX-rates.

The promising results from the generalized tree structured model
further suggest the importance of modeling the stochastic process of
FX-rates by incorporating different regimes based on the underlying
macroeconomic situation. This result, together with the estimation of
parameters in line with the reversal property that guarantees the long
term stable relation among macroeconomic fundamentals and FX-rates
might suggest how the adjustment process to the stable long run
equilibrium is rather unstable through time and requires therefore a
sluggish adjustment process as the one of alternating regimes rather
than a gradual, periodic, adjustment as the one modeled by the
vector error correction model.

Despite, the above achievements some structural flows remain in the
analysis. Starting, with a theoretical notice, it should be mentioned
how all of the four currencies analyzed are considered to be safe haven
currencies in the broad academic literature. This, implies that
important movements for the series might be affected by important
political events as documented by Ranaldo and S{\"o}derlind (2010,
\cite{RanaldoSoderlind}) rather than by macroeconomic fundamental
movements leading to a worsen performance of the structural monetary
fit. Interesting would be in this sense to analyze the results for
other less liquid FX-rates series and observe whether the results are
in line with the one presented in the analysis as well as in the
benchmark papers.

Moreover, some general issues remain with the macroeconomic proxies
used in the analysis. Figure \ref{fig:macro} shows how the
unemployment rate and the gross domestic product growth are generally
negatively correlated supporting the validity of proxy of choice,
nonetheless, the possibility of further supporting the results by
leveraging a linear interpolation of the GDP level or the crude oil
consumption level subsist. 

Finally, albeit the generalized tree structured model well managed to
significantly outperform a random walk model, it should be further
stressed the different information set leveraged by the model
suggesting caution in claiming the superiority of the
generalized tree structured model for modeling FX-rates
out-of-sample.