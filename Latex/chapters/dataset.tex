\section{Dataset} \label{sec:dataset}

The dataset, comprehensively discussed in Appendix A \ref{app:A}
consists of monthly observations over 21 years ranging from January
1986 to January 2006.  All the time series of use are selected in
accordance with the underlying monetary structural model
\ref{eq:structural} and are consistent with the series first utilized
by Meese and Rogoff (1983a, \cite{MeeseRogoffa}). Three different
FX-rates series are analyzed, namely the JPN/USD rate, the GBP/USD
rate and the CHF/USD rate.  With respect to the independent variables
of the structural model, I decided to work with the 3-months treasury
bills to capture the short term interest rates differentials. The only
exception were of CH series where I worked with the 3-month US-CH
LIBOR spread due to missing reported publicly available data for the
seasonally unadjusted short term treasury rates at monthly lag.  \\
For measuring money mass differentials, I worked with the M1 and M3
measures testing the both given the important difference among the two
and the lack of theoretical models to favour one measure above the
other. I obtained stronger results for the M3 series and therefore
proceeded to report the results obtained through such measure in the
paper. \\ For measuring the inflation rate differentials important for
the Dorbusch model I worked with consumer price indices and for
measuring the trade balance I worked with the net trade amount for
goods as a proxy for the current account balance.  The trade balance
is indeed just a rough approximation given the fraction of traded
goods in comparison to the total amount of trade that includes
services and monetary transfers. Nonetheless I preferred such measure
from an interpolation of the quarterly published current account balances. \\
Finally, I worked with the unemployment rate to capture the output
differential of the economies as such variable reflects
well the fluctuations in output levels, i.e. the quantitiy of
interest, leaving the constant term in equation \ref{eq:structural} to
capture the other information present in the output measure such as
the size of the economies.

As in our reference papers I decided not to work with seasonal time
series in order to avoid the possible bias introduced by different
seasonal adjustments to the structural parameters documented in Sims
(1973a and 1973b, \cite{Simsa, Simsb}).  Henceforth I decided to
approach our model validation by fitting the models on seasonal
adjusted and detrended time series.

With respect to the time series adjustments I proceeded by detrending
the series exploring three different possibilities. Firstly,
detrending via differentiation, secondly detrending through a linear
time trend and thirdly detreding through a moving average
filter. Detrending through differencing proved to be the most
effective due to the quadratic behaviour displayed by the series. This
approach was therefore applied to all of the series.  The obtained
mean stationary series were consequently inspected for the presence of
seasonality. This was done using spectral peaks derived from the
fast-Fourier transform and the correspondigly power density to detect
seasonality in accordance with Nerlove (1964, \cite{Nerlove}). The two
most common seasonalities in the stationary times series were
subsequently captured by modeling fourier terms, where the number of
fourier terms was chosen based on the standard Akaike criteria of
\ref{eq:AICc}. New seasonal adjusted times series were finally
computed by factoring the modeled seasonality out thorough the
following form:

\begin{equation} \label{eq:seasonal adjustment}
  y^{adjusted} = y^{unadjusted} - \sum_{k=1}^K [ \alpha_k \sin(\frac{2 \pi kt}{m}) + \beta_k \cos(\frac{2 \pi kt}{m}) ] - \sum_{j=1}^J [ \alpha_j \sin(\frac{2 \pi jt}{m}) + \beta_j \cos(\frac{2 \pi jt}{m}) ]
\end{equation}

where $K$ and $J$ represent the optimal number of fourier terms to
include to capture the two seasonalities according to the AIC
information criteria and $a_k, a_j, b_k, b_j$ the optimal estimated
coefficients to model the seasonality in the series.

Finally the newly generated seasonal adjusted series were tested for
seasonality according to the seasonal unit roots tests presented by
Canova and Hansen (1995, \cite{CanovaHansen}) and Wang and Smith
(2006, \cite{WangSmith}) and the process above reitered until no
evidence of seasonality could be found for any of the seasonal lags
suggested by spectral peaks.
