\section{Dataset} \label{sec:dataset}

The dataset, comprehensively discussed in Appendix A \ref{app:A} consists of
monthly observations for a 21 years time frame ranging
from January 1986 to January 2006.
All the times series of use are selected in accordance with the underlying monetary
structural model \ref{eq:structural} and are consistent with the series first utilized by
Meese and Rogoff (1983a, \cite{MeeseRogoffa}). Three different FX-rates series will be
analyzed, namely the JPN/USD rate, the GBP/USD rate and the CHF/USD rate.
With respect to the independent variables of the structural model, we decided to work
with the 3-months treasury bills to capture the short term interest rates differentials
with the only exception of CH series where we worked with the 3-month US-CH LIBOR spread
because of missing reported public available data for the seasonally unadjusted short term treasury rates at monthly lag.
\\ For measuring money mass differentials
we worked using the M1 and M3 measures testing the both given the important difference among the two 
and the lack of theoretical models to favour one measure above the other. We obtained stronger results for the M3 series
and therefore decided to report the results obtained trough such measure in the paper. \\ For measuring the
inflation rate differentials important for the Dorbusch model we worked using the consumer price indeces and
for measuring the trade balance we worked using the net trade amount for goods as a proxy for the current account balance.
The latter is of course just a rough approximation
given the fraction of traded goods in comparison to the total amount of trade including services and monetary transfers,
nonetheless we preferred the latter measure to an interpolation of the quarterly published current account balances. \\
Finally, we worked with the unemployment rate to capture the output differential of the economies as we
believe that such variable
well manages to capture the fluctuations in output levels, the quantitiy of interest, leaving
the constant term in equation \ref{eq:structural} to capture the other information present
in the output measure such as the size of the economies.

As in our reference papers we decided to work using not seasonally times series in order to avoid
the possible bias introduced by different seasonal 
adjustments on the structural parameters documented in Sims (1973a and 1973b, \cite{Simsa, Simsb}).
Henceforth we decided to approach our model validation in two different ways fitting the models
on seasonal adjusted and detrended times series and on untreated times series in levels.
This will of course have important consequences for the interpretation of the model results.
While detrended and seasonally adjusted stationary times series will allow an interpretability
of the results the latter is not guaranteed for the not treated times series where spurious regressions
might arise distorting the model interpretation. Given the fact, that the goal of this work
is nonetheless a reliable forecast of FX-rates rather than the identification of the causal
relation between macroeconomic variables and FX-rates an analysis of the series in levels is
ultimately interesting for the analysis and sometimes resulted in superior results in comparison to treated times
series as in Meese and Rogoff (1983a, \cite{MeeseRogoffa}).

With respect to the times series adjustments we proceeded by detrending the series at first exploring three
different possibilities. Firstly detrending via differentiation, secondly detrending through a
linear time trend and thirdly detreding through a moving average filter. Detrending through differencing
provided to be the most effective due to the quadratic behaviour displayed by
the series. This approach was therefore applied to all of the series.
The obtained mean stationary series were consequently inspected for the presence of seasonality and appropriately
differenciated to the point where no significance was found for seasonal units roots according to the methods
proposed in Canova and Hansen (1995, \cite{CanovaHansen}) and in Wang and Smith (2006, \cite{WangSmith}).