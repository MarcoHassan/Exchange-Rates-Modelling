\section{Dataset}
\label{sec:part1}

The dataset, comprehensively discussed in Appendix A \ref{app:A} consists of
monthly observations for a 21 years time frame ranging
from January 1986 to January 2007.
All the times series of use are selected from the FRED database
in accordance with the underlying monetary
structural models and are consistent with the series first utilized by
Meese and Rogoff (1983a, \cite{MeeseRogoffa}).
As in our reference papers we decided to work using not seasonally
adjusted times series in order to avoid the possible bias introduced by different seasonal
adjustments on the structural parameters documented in Sims (1973a and 1973b, \cite{Simsa, Simsb}).
Henceforth we decided to approach our model validation in two different ways fitting the models
on seasonal adjusted and detrended times series and on untreated times series in levels.
This will of course have important consequences for the interpretation of the model results.
While detrended and seasonally adjusted stationary times series will allow an interpretability
of the results the latter is not guaranteed for the not treated times series where spurious regressions
might arise distorting the model interpretation. Given the fact, that the goal of this work
is nonetheless a reliable forecast of FX-rates rather than the identification of the causal
relation between macroeconomic variables and FX-rates an analysis of the series in levels is
ultimately interesting for the analysis and will yield superior results in comparison to treated times
series as in Meese and Rogoff (1983a, \cite{MeeseRogoffa}).
With respect to the times series adjustments we proceeded by detrending the series at first exploring three
different possibilities. Firstly detrending via differentiation, secondly detrending through a
linear time trend and thirdly detreding through a moving average filter. Detrending through differencing
provided to be the most effective compared to linear detrending due to the quadratic behaviour displayed by
the series. This approach was therefore applied to all of the series but the current account balance
differentials, where the particular monotonic behaviour of the series required a moving average
filter able to separate the rapidly fluctuating component of the series from the slow varying component.
The obtained mean stationary series were consequently inspected for the presence of seasonality and appropriately
differenciated to the point where no significance was found for seasonal units roots according to the methods
proposed in Canova and Hansen (1995, \cite{CanovaHansen}) and in Wang and Smith (2006, \cite{WangSmith}).