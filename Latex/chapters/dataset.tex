\section{Dataset} \label{sec:dataset}

The dataset, comprehensively discussed in Appendix A \ref{app:A} consists of
monthly observations over 21 years ranging
from January 1986 to January 2006.
All the time series of use are selected in accordance with the underlying monetary
structural model \ref{eq:structural} and are consistent with the series first utilized by
Meese and Rogoff (1983a, \cite{MeeseRogoffa}). Three different FX-rates series are
analyzed, namely the JPN/USD rate, the GBP/USD rate and the CHF/USD rate.
With respect to the independent variables of the structural model, we decided to work
with the 3-months treasury bills to capture the short term interest rates differentials. The only exception were of CH series where we worked with the 3-month US-CH LIBOR spread
due to missing reported publicly available data for the seasonally unadjusted short term treasury rates at monthly lag.
\\ For measuring money mass differentials,
we worked with the M1 and M3 measures testing the both given the important difference among the two 
and the lack of theoretical models to favour one measure above the other. We obtained stronger results for the M3 series
and therefore proceeded to report the results obtained through such measure in the paper. \\ For measuring the
inflation rate differentials important for the Dorbusch model we worked with consumer price indices and
for measuring the trade balance we worked with the net trade amount for goods as a proxy for the current account balance.
This trade balance is indeed just a rough approximation
given the fraction of traded goods in comparison to the total amount of trade that includes services and monetary transfers.
Nonetheless we preferred such measure from an interpolation of the quarterly published current account balances. \\
Finally, we worked with the unemployment rate to capture the output differential of the economies as we
believe that such variable
reflects well the fluctuations in output levels, i.e. the quantitiy of interest, leaving
the constant term in equation \ref{eq:structural} to capture the other information present
in the output measure such as the size of the economies.

As in our reference papers we decided not to work with seasonal time series in order to avoid
the possible bias introduced by different seasonal 
adjustments to the structural parameters documented in Sims (1973a and 1973b, \cite{Simsa, Simsb}).
Henceforth we decided to approach our model validation by fitting the
models on seasonal adjusted and detrended time series.

With respect to the time series adjustments we proceeded by detrending the series exploring three
different possibilities. Firstly, detrending via differentiation, secondly detrending through a
linear time trend and thirdly detreding through a moving average filter. Detrending through differencing
proved to be the most effective due to the quadratic behaviour displayed by
the series. This approach was therefore applied to all of the series.
The obtained mean stationary series were consequently inspected for the presence of seasonality and appropriately
differentiated to the point where no significance was found for seasonal unit roots according to the methods
proposed in Canova and Hansen (1995, \cite{CanovaHansen}) and in Wang and Smith (2006, \cite{WangSmith}).