\section{Methodology}
\label{sec:part2}

\subsection{Structural Model}

The basic structural model encompassing all of the different monetary models
discussed in the introductory session is of the following form:

\begin{equation} \label{eq:structural}
  s = \beta_{0} + \beta_{1}(m - m^{*}) + \beta_{2}(y-y^{*}) + \beta_{3}(r-r^{*})
  + \beta_{4}(\pi - \pi^{*}) + \beta_{5}(TB - TB^{*}) + \varepsilon
\end{equation}

where $s$ represents the logarithm of the indirect quote of FX-rates, i.e. the
foreign exchange value of a dollar unit, $m-m^{*}$ represents the logarithm of
the U.S. and foreign country money mass supply differential, $y-y^{*}$ represents the logarithm
of the U.S. and foreign GDP level, $r-r^{*}$ represents the short term U.S. forign country
interest rate differential, $\pi - \pi^{*}$ represents the U.S. and foreign country inflation
rate differential, $TB - TB^{*}$ represents the U.S. foreign country current account differential
normalized to be one at the beginning of the sample period and, finally, where
$\varepsilon$ represents the error term of the regression capturing all of the
other factors not expressed in the model.\\
The model above encompasses all of the monetary models discussed. In the specific
the most basics monetary model, the Frenkel-Bilson model logically inferred from
the relative PPP proposition assumes $\beta_{4} = \beta_{5} = 0$. The Dornbusch model allowing
a sluggish price adjustment behaviour and predicting FX-overshooting
assumes $\beta_{5} = 0$ and finally the Hooper-Morton model poses no restriction on
the coeffcients of equation \ref{eq:structural}. Such a structural model will consequently
be estimated on four different parametric models of interest, which will be discussed next.

\subsection{Univariate Models}
\label{sub:univariate}

The most basic model that will be tested is an OLS model as in the benchmark papers of Meese
and Rogoff (1983a, 1983b, \cite{MeeseRogoffa, MeeseRogoffb}). In this basic first model the
coefficients of equation \ref{eq:structural} will be computed without looking at any lagged effect.
In contrast to it a second more flexible model will be applied to capture possible lagged effects
of the macroeconomics fundamentals. In comparison to the Meese and Rogoff papers where the authors decided
to capture the possibility of lagged terms by incorporating autoregressive models giving a higher
importance on more recent observations we decided to apply the transfer function models widely spread in
the fields of engineering such as control systems and electronic circuits. These models have been poorly
discussed in the field of economics with the exception of Tustin (1957, \cite{Tustin}) that tried to
make the point for applying the models into the economics modeling field.\\
Transfer function models relates a given set of inputs to an output variable through the following general formula

\begin{equation} \label{eq:transfer}
  Y_{t} = \mu + \frac{(\omega_{0}+\omega_{1}B^{1}+\dots+\omega_{s}B^{s})}{1-\delta_{1}B^{1}-\dots-\delta_{r}B^{r}}X_{t-b} + \varepsilon_{t}
\end{equation}

where $X$ represent a matrix of exogenous terms, $\mu$ the optional term modeling the mean of the series and
$\varepsilon$ the non-captured variation in the series. Moreover the above ratio represents the transform
function of the regressors matrix and is especially caracterized by the order of the denominator and
nominator terms. $r$, the order of the denominator term, expresses the rate of the decay pattern, where
a higher term indicates a slower decay.  $s$, the order of the nominator term, expresses the persistence of
so called unpatterned spikes, that is the persistence of effects that are not captured in the decay pattern.
Finally, the $b$ teerm in the matrix of input represent the dead time, that is the time it takes for the dependent
variable to react to some changes in the input matrix. This is of primary importance as it might very well be
that some shift in macroeconomics fundamentals just start to display effects after a certain amount of time when
the economics actors start to perceive the change.\\
Due to the flexible nature of the latter model, we believe that it is better suited and preferrable to
capture the true nature of distributed lags present in the structural model of equation \ref{eq:structural} in
comparison with the smoothed autoregressive model applied by Meese and Rogoff with the arbitrarly chosen smoothing
term of 0.95 (See Meese and Rogoff (1983a pp. 7, \cite{MeeseRogoffa})).


\subsection{Multivariate Models}
\label{sub:multivariate}
The above discussed models rely on the exogeneity of the independent variables. If the assumption
fails the model will be biased and would lead to misleading conclusion as the independence of the
sample distribution would not be guaranteed. In practice this poses an issue for the estimation
of the structural model of equation \ref{eq:structural}. While some variables as the monetary
mass and the output gap are commonly treated as exogenous variables in the underlying monetary models
the practice suggest that such macroeconomic variables might very well be influenced by the
movements of FX-rates. On the top of it other variables such as short term interest rates differentials
are treated as endogenous even in the underlying monetary models and therefore require a different estimation
compared to the one outlined in the models of the previous section.\\
To obviate the above issues Meese and Rogoff (1983a and 1983b, \cite{MeeseRogoffa, MeeseRogoffb}) analyzed the structural model
of interest through a vector autoregressive model firstly introduced by Sims (1980, \cite{Sims1980}).
In the general case the model consists of a system of equations of the form

\begin{equation} \label{eq:VAR}
y_{t} =  \Phi_{0}+ \Phi_{1}y_{t-1} + \Phi_{2}y_{t-2} + \dots + \Phi_{n}y_{t-n} + \varepsilon_{t}
\end{equation}

where $y$ is a vector containing the endogenous variables of interest, $\Phi_0$ is a vector of constants,
$\Phi_1,  \dots  , \Phi_n$ are matrices describing the effect of lagged endogenous variables on the levels of
the current variables and $\varepsilon$ captures the equations specific error term.
The resulting model will capture the endogeneity present in the monetary structural model allowing a consistent
OLS estimation as far as the error terms of the equations will be uncorrelated.

Despite the described vector autoregressive model well manages to model the endogeneity of macroeconomic
variables we question whether a restricted form of it could yield more efficient and reliable estimates capturing
the FX-rates and the monetary models relation. This is especially motivated by the cointegration theory developed
by Engle and Granger (1987, \cite{EngleGranger}) and the well known evidence of non stationary macroeconomics
times series \footnote{See for instance Gil-Alana and Robinson (1997, \cite{GilAlanaRobinson})}. Moreover, the impportance
of such mehtod is underlined by evidence of Phillips (1986, \cite{Phillips}) that set down the theoretical fundamentals
showing how parameter estimates of cointegrated series will not converge in probability and will not converge to any
non-degenerate distribution in the asymptotic case if the case of a misspecified OLS estimate as potentially is equation \ref{eq:VAR}.

We propose therefore a test for cointegration among the macroeconomics and the FX-rates series
based on Johansen (1991, \cite{Johansen}) %% scrivere Johansen procedure ??%%
and we consequently estimate a vector error correction model of the form

\begin{align}
  \Delta y_t =& \ \Pi y_{t-1} + \sum^{p-1}_{i=1} \Phi^{*} \Delta y_{t-i} + \varepsilon_t \nonumber\\
  \Phi^{*} =& \ - \sum^{p}_{i=j+1} \Phi_i, \ \  j = 1, \dots, p-1 \label{eq:VECM}\\
  \Pi =& \ -(I - \Phi_1 - \dots - \Phi_p) \nonumber\
\end{align}

where $\Pi y_{t-1}$ of equation \ref{eq:VECM} represent the error correction term and $y_i$ and $\Phi$
refer to the variables described in the unrestricted vector autoregressive model of \ref{eq:VAR}.

\subsection{Forecasting Approach}
\label{sub:forecast}
The four described models described and a random walk model without drift will be validated by looking
at their ability to forecast FX-rates out of sample. In this sense the 234 observations sample
is splitted in a training and validation sample. Three fourth of the total observations will
be used for training the four models outlined above and the rest of the observations will be used for the
out of sample validation.

In comparison to Meese and Rogoff
(1983b, \cite{MeeseRogoffb}) we will not try any restricted estimation based on the theoretical
monetary models literature but will rather estimate unrestricted versions for all of the models
outlined with the exception of the vector error correction model described in \ref{eq:VECM}, given
the by product VAR model restriction imposed by the latter.\\
All of the four different models will be estimated according to the most general structural model
described in \ref{eq:structural} without restrictions. Based on the structural model results
Wald tests on the restricted structural model will be computed in order to check whether there
is in sample evidence to consider restricted monetary models. 

Based on the first estimation we proceeded by calculating the out of sample performance of the
different models applying a rolling forecast in analogy to the benchmark papers. This will consist
of a reestimation of the four outlined models for each new forecasting point. Given the decision
to estimate the out of sample model performance at one, three, six and twelve months lags the rolling
forecast technique involves a model reestimation at such frequencies. Important is nonetheless to underline how
the results of the Wald tests will be extended for each of the subsequent model estimation
in the rolling forecast. This means that given statistical significant evidence for the Null of a
restricted model in the first three fourth of the sample the same restricted model will be used for the
subsequent model estimations. \\
Explicitely this method consists to fit a structural model containing all of the macroeconomics
series in the first 176 observations, that is from September 1986 to April 2001, and perform Wald
tests to check whether a restricted monetary model is supported in sample. Based on such restults
the rolling forecast method involves first to estimate the next point forecast at 1, 3, 6 and 12 months
and in a second step to shift the data sample of one period such that a new model will be estimated for
the October 1986 to May 2001 series and, according to the new parametric fit,
point forecast at 1, 3, 6 and 12 months will be estimated. This method will then be iterated until the last out of sample
observation for February 2006 is reached.

Finally, as in the benchmark papers, we will allow the univariate models
described in \ref{sub:univariate} a richer set of information compared to the random walk.
In the specific the random walk and the multivariate model will use the $\mathscr{F}_{t-1} = {F_0, \ \dots, \ F_{t-1}}$
information set, where $F_i$ represent a set containing all of the available information at timepoint $i$.
By contrast the univariate models will dispose of $\mathscr{F}_t \setminus s_t$, where $s_t$ represents
the FX-rate at time point $t$.
In simple terms this means that we are going to give the univariate models the advantage of using the actual realizations of
macroeconomics variables without the need of estimating them.

Based on the obtained rolling forecasts three statistics will be computed to compare the model fit.
These are the root mean squared error (RMSE), the mean absolute error (MAE) and the mean directional accuracy (MDA).
A particular importance will be assigned to the MAE results given Westerfield (1977, \cite{Westerfield})
that analyzed the empirical exchange rates distribution finding evidence for FX-rates
non-normal stable-Paretian distributions with infinite variance.\\
In contrast with Meese and Rogoff (1983a and 1983b, \cite{MeeseRogoffa, MeeseRogoffb}) we decided
to further explore the point estimators of the above mentioned statistics by computing a model confidence set (MCS)
as described in Hansen et al. (2011, \cite{HansenMCS}).
The idea of the latter consists of a procedure yielding a model set, $\mathcal{M}^*$, built to contain the
best model with a chosen level of confidence. The exact procedure is based on an equivalence test $\delta_m$ and an
elimination rule $\epsilon_m$, consistent with the chosen confidence level. In a first step the competing models will be
compared with each other. If $\delta_m$ does not support evidence for the equal performance of the models, $\epsilon_m$
is applied and the poorly performing models are discarded and the general problem reiterated until $\delta_m$ is accepted
for all of the surviving models.\\
In this paper we decided to make use of superior predictive ability test of Hansen (2005, \cite{HansenSPA})
and to obtain p-values for the equal predictive hypothesis of models according to a bootstrap implementation
outlined by Hansen et al. (2011, \cite{HansenMCS}).