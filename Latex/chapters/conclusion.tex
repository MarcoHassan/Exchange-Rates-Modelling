\section{Conclusion}
\label{sec:conclusion}

The reported results generally confirmed the conclusions reported in Meese and Rogoff (1983a and 1983b, \cite{MeeseRogoffa, MeeseRogoffb}).
No one of the four models analyzed could statistically significantly improve the out of sample performance of a simple random walk 
model without drift when forecasting of foreign exchange rates. Moreover, we found evidence for a statistically significant underperformance of
OLS models compared to a random walk model.  

When looking at the point estimation of out of sample error statistics we observed a mixed evidence for the vector autoregressive models and the
transfer function models. Both models could sporadically yield more accurate results compared to the random walk model but generally underperformed
the latter.

Given the evidence for cointegration among macroeconomics variables and FX-rates we estimated a vector error correction model which displayed promising
results outperforming the performance of a random walk model when looking at out of sample forecasting error statistics. These results were consistent
among all estimation lags and country series but nonetheless did not display enough evidence for rejecting the Null of equal predictability with the
random walk model based on the interval estimation through the model confidence test of Hansen (2011, \cite{HansenMCS}).

It is possible to conclude therefore by underlying that our study does support the strength of a random walk in comparison to univariate models
for FX-rates estimation and as a general sound model for capturing the unit root behaviour observed in the FX-rates movements. Nonetheless, our study
vastly contrasted with the results of Meese and Rogoff papers when analyzing multivariate models. The latter models were able to well capture the
movements of FX-rates through a parametric model based on macroeconomic fundamentals suggesting an estimation failure due to a small sample bias
in the results reported by Meese and Rogoff (1983a and 1983b, \cite{MeeseRogoffa, MeeseRogoffb}). Moreover the cointegration between macroeconomics
fundamentals and FX-rates suggests that a long run stable relation between the two exists.

Based on our results we reject the general claim of weak relation among macroeconomics variables and foreign exchange rates and we claim that the
latter should be considered as a primary determining factor driving the foreign exchange rates movements at monthly lag. If the well performance of the random walk
models is explained by the long memory displayed by foreign exchange series that is visibile by investing autocorrelation plots for the series
in virtually all time periods, the macroeconomic models managed a comparable performance simply leveraging the movements, long run relation and endogeineity
among foreign exchange and macroeconomic variables and require therefore a stronger attention from the general academic literature.