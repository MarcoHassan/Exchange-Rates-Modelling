\section{Conclusion}
\label{sec:conclusion}

The reported results generally confirmed the conclusions presented by Meese and Rogoff (1983a and 1983b, \cite{MeeseRogoffa, MeeseRogoffb}).
None of the four models analyzed could improve with statistical significance the out of sample performance of a simple random walk 
model without drift when forecasting foreign exchange rates. Moreover, we found evidence for a statistically significant underperformance of
OLS models compared to a random walk model.  

When looking at the point estimation of out of sample error statistics we observed a mixed evidence for the vector autoregressive models and the
transfer function models. Both models could sporadically yield more accurate results compared to the random walk model but generally underperformed
the latter.

Given the evidence for cointegration among macroeconomic variables and FX-rates we estimated a vector error correction model which displayed promising
results, outperforming a random walk model when looking at out of sample forecasting error statistics. These results were consistent
among all estimation lags and country series. Nonetheless they did not display enough evidence for rejecting the Null of equal predictability with a
random walk model based on the interval estimation through the model confidence test of Hansen (2011, \cite{HansenMCS}).

Therefore, we can conclude that our study does support the strength of a random walk model in comparison to univariate models
for FX-rates estimation. It is also generally sound model for capturing the unit root behaviour observed in the FX-rates movements. Nonetheless, our study
vastly contrasted with the results of Meese and Rogoff papers when analyzing multivariate models. The latter models were able to well capture the
movements of FX-rates through a parametric model based on macroeconomic fundamentals suggesting an estimation failure due to a small sample bias
in the results reported by Meese and Rogoff (1983a and 1983b, \cite{MeeseRogoffa, MeeseRogoffb}). Moreover the cointegration between macroeconomic
fundamentals and FX-rates suggests that a stable long run relation between the two exists.

Based on our results we reject the claim of weak relation among macroeconomic variables and foreign exchange rates and we claim that the
macroeconomic fundamentals should be considered as a primary determining factor driving the foreign exchange rates movements at monthly lag.

Finally, we note how the successful performance of the random walk
models is explained by the long memory displayed by foreign exchange series that is visibile by analyzing autocorrelation plots for the series
in virtually all time periods.
In contrast to it, the comparable performance of  macroeconomic models is explained by the long run relation and correlation among foreign exchange and macroeconomic variables and FX-rates. This is a remarkable achievement that leads us to claim for an underlying relation between fundamentals and FX-rates rejecting the exchange rate disconnect puzzle for the studied period.
