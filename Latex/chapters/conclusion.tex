\section{Conclusion}
\label{sec:conclusion}

This thesis proposed a revaluation of the exchange rate disconnect
international finance puzzle, claiming a weak realtion between
macroeconomic fundamentals and FX-rates.

Given the small sample of use in the papers of Meese and Rogoff (1983a
and 1983b, \cite{MeeseRogoffa, MeeseRogoffb}) \textendash \ the reference in the
academic literature \textendash \ I created a data set qualitatively comparable to
the one of the benchmark papers encompassing roughly three times as
many observations. Based on the latter, a re-estimation of univariate
and multivariate models in analogy to the benchmark papers was proposed.

The reported results generally confirm the conclusions presented by
Meese and Rogoff (1983a and 1983b, \cite{MeeseRogoffa, MeeseRogoffb}).
None of the four univariate and multivariate models analyzed could
improve with statistical significance the out of sample performance of
a simple random walk model without a drift when forecasting foreign
exchange rates. Moreover, I found evidence for a statistically
significant underperformance of the OLS model without inclusion of
lagged terms in comparison to a random walk model.

When looking at the point estimation of the out of sample error
statistics I observed a mixed evidence for the vector autoregressive
model and the transfer function model. Both models could sporadically
yield more accurate results compared to the random walk model but
generally underperformed the latter.

Given the evidence for cointegration among macroeconomic variables and
FX-rates I estimated a vector error correction model which displayed
promising results by outperforming a random walk model when looking at
the out of sample forecasting error statistics. These results were
consistent among all estimation lags and country series. Nonetheless
the results did not display enough evidence for rejecting the Null of
equal predictability with a random walk model based on the interval
estimation through the model confidence test of Hansen (2011,
\cite{HansenMCS}). Moreover, it is interesting to notice that the
results presented contrast with the two reference papers testing for
co-integration among macroeconomic fundamentals and FX-rates, namely
Baillie Selover (1987, \cite{BaillieSelover}) and Meese and Rose
(1991, \cite{MeeseRose}). Both of the papers reported no evidence for
co-integration among the series for the period 1974-1987 and therefore
did not proceed with an error correction modeling. This opens the
discussion of possible causes for the changed evidence. Given the
papers' use of the two step Engle and Granger approach (1987,
\cite{EngleGranger}),combining a first OLS estimation with an
augmented dickey fuller test on the residuals, the source of the
difference might not lie in a small sample bias. Rather, it is
possible that the longer period analyzed in the thesis is more robust
to major political events in comparison to the sample period used by
the papers, where for instance the crude oil crises of the 1970s might
have disrupted the long term macro-FX relation.

If the vector error correction model could not display a
statistically significant performance increase in comparison to the
random walk model, the generalized tree structured model did. In line
with Hisieh (1989, \cite{Hisieh}), Domowitz and Hakkio (1985,
\cite{DomowitzHakkio}) and Diebold and Pauly (1988,
\cite{DieboldPauly}), who documented important non-linearities in the
mean of FX-rates movements, and in line with Kugler and Kr{\"a}ger
(1993, \cite{KuglerKrager}) and Engle and Hamilton (1989,
\cite{EngleHamilton}) that documented the statistically significant
presence of regime shifts in the data generating process of FX-rates,
the pruned tree obtained by the generalized tree structured model
suggest the importance of modeling thresholds and regime shifts for
capturing FX-rates dynamics.

The obtained regimes suggest the presence of different regions with
different FX-rates changes dynamics.  Moreover, the interesting fact
that threshold variables coincide with the co-integrated macroeconomic
series together with local parametric estimates in line with stable
long run relation among macroeconmic fundamentals and FX-rates further
strengthen the centrality of treating macroeconomic and FX-rates
series together.  Finally, the sensible improve in the out of sample
performance of the generalized tree structured threshold model points
to an adjustment to the long term stable equilibria among
macroeconomic series and FX-rates occurring through sharp adjustments,
identifiable with the different regimes, rather than through a stable,
gradual, adjustment such as the one proposed by the vector error
correction model.

Based on these results I reject the claim of weak relation among
macroeconomic variables and foreign exchange rates and I claim that
the macroeconomic fundamentals should be considered as a primary
determining factor driving the foreign exchange rates movements at
monthly lag.

Finally, I note how the successful performance of the random walk
models is explained by the long memory displayed by foreign exchange
series that is visibile by analyzing autocorrelation plots for the
series in virtually all time periods.  In contrast to it, the
comparable performance of macroeconomic models is explained by the
long run the sole relation and correlation among foreign exchange
rates and macroeconomic variables. This is a remarkable achievement
that leads me to claim for an underlying relation between fundamentals
and FX-rates rejecting the exchange rate disconnect puzzle for the
studied period.
