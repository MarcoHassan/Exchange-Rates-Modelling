\section{Introduction}

%First part - Issue of interest for your paper
%
%talk about monetary models and their failure. Cite Meese and Rogoff
%
Many approaches tried to model FX-rates since the start of a free-floating system after Bretton Woods in the 1970s. Monetary models, linking macroeconomics variables to FX-rates, were the first to emerge and due to their compelling theoretical intuition soon became the reference in the field. The enthusiasm for such elegant models was nonetheless soon diminished by two seminal works. Meese and Rogoff (1983a, \cite{MeeseRogoffa}) and Meese and Rogoff (1983b \cite{MeeseRogoffb}) demonstrated how no one of the different monetary models could beat a random walk without a drift in modeling FX-rates out of sample. After the evidence was presented the opinion divided. Some argued for a sample issue and confined the problem to the specific historical period, some argued for the limit of the linear models used by the authors to approach monetary models, some argued for a fail of monetary models and their underlying assumptions and finally some started to approach the issue in an innovative way turning their interest towards micro-based models aiming to capture the complexity of market information asymmetries and investors heterogeneity among the others \footnote{See Balliu and King (2005, \cite{BalliuKing}) for a general survey.}

% Second Part - What we do in a nutshell
While all such different reactions contributed to a further understanding of FX-rates properties we question whether the runaway from linear monetary models is justified given the methodological approach presented in the studies of Meese and Rogoff and the lack of a cointegration analysis later formalized by Granger and Engle (1987,  \cite{EngleGranger}). Given the gap in academic literature exploring this cointegration relationship our study will propose a study of it at monthly lag.  %
%Third Part - Motivation. Why is it important? Who cares?
% 
% Avoid overengineering and overcomplication. Provide simple setting
% Understand if macroeconomics variables are cointegrated at monthly lag frequency.
%
This methodological study stands therefore as a revision of the out of sample failure of foreign exchange rates testing whether macroeconomics variables are indeed incapable of explaining FX-rates movements before exploring more nuanced fields such as nonlinear and micro-based modeling. The hope is therefore to avoid over engineering and over complication of the issue going back at the fundamentals providing market makers and policy makers a simple framework to work with.

%Fourth Part - What we do in more details
In the specific, our study aims at exploring the above at a monthly lag frequency on a self developed data set for the recent period ranging from 1986 to 2007. Such period will allow a consistent estimation of all of the necessary parameters given the models of choice, moreover it will avoid to model the most recent financial crises where structural breaks in the series arise.
As in Meese and Rogoff (1983a and 1983b, \cite{MeeseRogoffa,MeeseRogoffb}) we are going to explore the out of sample forecasting of foreign exchange rates for three major monetary models. Firstly, the monetary models first presented by Frenkel (1976, \cite{Frenkel1976}) and  Mussa\ (1976, \cite{Mussa1976}) claiming for a link among monetary mass differentials, interest rates differentials output gap and FX-rates. Secondly the sticky price model where inflation rates differential plays a crucial role as prices adjusts sluggish to macroeconomics shocks affecting FX-rates (Dornbush 1976, \cite{Dornbusch}). Finally, a portfolio balance model claiming for the importance of monetary flows and net current account differentials as a key determinant for FX-rates changes as in Hooper and Morton (1982, \cite{HooperMorton1982}). \\
Based on such monetary models five different models will be analyzed in order to forecast FX-rates. Three of them are presented in the seminal work of Meese and Rogoff (1983a, \cite{MeeseRogoffa}) and will pose the benchmark to check whether the result of the paper hold in the time frame of interest. These are a simple random walk forecast, an OLS estimation without inclusion of lagged terms and a vector autoregressive model. On the top of it we will analyze the performance of transfer models discussed in Montgomery and Weatherby (1980, \cite{MontgomeryWeatherby}) and the vector error correction model representation of the cointegration relationship between macroeconomics variables and FX-rates proposed by Granger and Engle (1987,  \cite{EngleGranger}).

%Fifth Part - Main Findings

%% TO BE ADDED

%Sixth Part - Contribution to the literature.
%% skip already incorporated above. 

%Seventh part - This paper is organized as follows
