\section{Introduction}

%First part - Issue of interest for your paper
%
%talk about monetary models and their failure. Cite Meese and Rogoff
%
Many approaches tried to model FX-rates since the start of a free-floating system after Bretton Woods in the 1970s. Monetary models, linking macroeconomics variables to FX-rates were the first to emerge and due to their compelling theoretical intuition soon became the reference in the field. The enthusiasm for such elegant models was nonetheless diminished by two seminal works. Meese and Rogoff (1983a, \cite{MeeseRogoffa}) and Meese and Rogoff (1983b \cite{MeeseRogoffb}) demonstrated how no one of the different monetary models could beat a random walk without a drift in modeling FX-rates out of sample. After the evidence was presented the opinion divided. Some argued for a sample issue and confined the problem to the specific historical period, some argued for the limit of the linear models used by the authors to approach monetary models, some argued for a fail of monetary models and their underlying assumptions and, finally, some started to approach the issue in an innovative way turning towards micro-based models aiming to capture the complexity of market information asymmetries and investors heterogeneity \footnote{See Balliu and King (2005, \cite{BalliuKing}) for a general survey.}.

% Second Part - What we do in a nutshell
While all such different reactions contributed to a further understanding of FX-rates properties we question whether the runaway from linear monetary models is justified given the methodological approach presented in the studies of Meese and Rogoff, the small sample of use and the lack of a cointegration analysis later formalized by Granger and Engle (1987,  \cite{EngleGranger}). 
%Third Part - Motivation. Why is it important? Who cares?
% 
% Avoid overengineering and overcomplication. Provide simple setting
% Understand if macroeconomics variables are cointegrated at monthly lag frequency.
%
This methodological study stands therefore as a revision of the out of sample failure of foreign exchange rates testing whether macroeconomics variables are indeed incapable of explaining FX-rates movements before exploring more nuanced fields such as nonlinear and micro-based modeling. The hope is therefore to avoid over engineering and over complication of the issue going back to the fundamentals.

%Fourth Part - What we do in more details
In the specific, our study aims at exploring the above at a monthly lag frequency on a self developed data set for the recent period ranging from 1986 to 2006. Such period will allow a consistent estimation of all of the necessary parameters given the models of choice, moreover it will avoid to model the most recent financial crises where structural breaks are present in the series.
As in Meese and Rogoff (1983a and 1983b, \cite{MeeseRogoffa,MeeseRogoffb}) we are going to explore the out of sample forecasting perfomance of foreign exchange rates for three major monetary models. Firstly, the monetary models first presented by Frenkel (1976, \cite{Frenkel1976}) and  Mussa\ (1976, \cite{Mussa1976}) claiming for a link among monetary mass differentials, interest rates differentials, output gap and FX-rates. Secondly the sticky price model where inflation rates differential plays a crucial role as prices adjusts sluggish to macroeconomics shocks (Dornbush 1976, \cite{Dornbusch}). Finally, a portfolio balance model claiming for the importance of monetary flows and net current account differentials as a key determinant for FX-rates changes as in Hooper and Morton (1982, \cite{HooperMorton1982}). \\
Based on such monetary models five different models will be analyzed in order to forecast FX-rates. Three of them are presented in the seminal work of Meese and Rogoff (1983a, \cite{MeeseRogoffa}) and will pose the benchmark to check whether the result of the paper hold in the time frame of interest. These are a simple random walk forecast, an OLS estimation without inclusion of lagged terms and a vector autoregressive model. On the top of it we will analyze the performance of transfer models discussed in Montgomery and Weatherby (1980, \cite{MontgomeryWeatherby}) and the vector error correction model representation of the cointegration relationship between macroeconomics variables and FX-rates proposed by Granger and Engle (1987,  \cite{EngleGranger}).

%Fifth Part - Main Findings
If our results are in line with the seminal papers of Meese and Rogoff in the case of univariate models the picture resulting from a multivariate fit
contradicts the results of the reference papers. We do not find enough evidence to support the claim of an underperformance of multivariate
linear models in comparison to a simple random walk model. Moreover we find strong evidence for cointegration among macroeconomics series and
FX-rates for all of the analyzed country series. This suggests a long term stable relation between the fundamentals and the FX-rates rejecting
the claim of Meese and Rogoff of a weak relation between the two. Finally, we do find that a vector error correction model representation of the cointegration
relation between macroeconomics series and FX-rates outperforms the out-of-sample forecast of FX-rates at all lags and for all of the country series further
strengthening the hypothesis that a link between monetary models and FX-rates does indeed exists at monthly lag frequency when the models are well behavied
and properly estimated.

%Seventh part - This paper is organized as follows
The paper continues as follows. Section 2 outlines the methodological approach and explains the various models and techniques used for
the empirical analysis and the out-of-sample forecast. Section 3 continues exploring the dataset and the chosen proxies to capture the
macroeconomics fundamentals of interest. Section 4 reports the main results of the empirical analysis and Section 5 concludes.