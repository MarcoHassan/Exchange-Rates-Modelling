\section{Introduction}

%First part - Issue of interest for your paper
%
%talk about monetary models and their failure. Cite Meese and Rogoff
%
Many approaches tried to model FX-rates since the start of a
free-floating system after Bretton Woods in the 1970s. Monetary
models, linking macroeconomics variables to FX-rates were the first to
emerge and due to their compelling theoretical intuition soon became
the reference in the field. Nonetheless, the enthusiasm for such
elegant models was diminished by two seminal works. Meese and Rogoff
(1983a, \cite{MeeseRogoffa}) and Meese and Rogoff (1983b,
\cite{MeeseRogoffb}) demonstrated how none of the different monetary
models could beat a random walk without a drift in modeling FX-rates
out of sample. After the evidence was presented the opinion
divided %%% Cite papers!!
into four distinct diverging groups. Some authors, such as Kugler and
Kr{\"a}ger (1993, \cite{KuglerKrager}), argued for a sample issue and
confined the problem to the specific historical period, other authors,
such as Lucas (1976, \cite{Lucas}), Engle and Hamilton (1989,
\cite{EngleHamilton}), Hsieh (1992, \cite{Hsieh}) and Chinn (1991,
\cite{Chinn}), argued for the limit of the linear models to approach
monetary models, other authors argued for the fail of monetary models
and their underlying PPP assumptions documented in Taylor and Taylor
(2004, \cite{TaylorTaylor}) and, finally, a last current of authors
such as Frankel, Galli, and Giovannini (1996,
\cite{FrenkelGalliGiovannini}), Lyons (2001, \cite{Lyons}) and Sarno
and Taylor (2001, \cite{SarnoTaylor}) started to approach the issue in
an innovative way turning towards micro-based models that aim to
capture the complexity of market information asymmetries and investors
heterogeneity.  \footnote{See Balliu and King (2005,
  \cite{BalliuKing}) for a general survey of FX-rates modeling
  in time.}


% Second Part - What we do in a nutshell
While all the above-mentioned different reactions contributed to a
further understanding of FX-rates properties I question whether the
runaway from linear monetary models is justified given the
methodological approach presented in the studies of Meese and Rogoff,
the small sample of use and the lack of a co-integration analysis
later formalized by Granger and Engle (1987,
\cite{EngleGranger}). Moreover, given the focus of the academic
literature on understanding the non-linear relation between
macroeconomics fundamentals and FX-rates by inferring alternating
conditional expectation optimal transformations introduced by Breiman
and Friedman (1985, \cite{BreimanFriedman}) and through the markov
switching models introduced by Hamilton (1989, \cite{Hamilton}), I
question whether a simple parametric model such as the generalized
tree structured model introduced by Audrino and B{\"u}hlmann (2001,
\cite{AudrinoBuhlmann}) allowing to capture both time dependent
structural breaks in the series such as the dot com bubble or the
Japanese bank crises of the 1990s as well as policy regime changes
identifiable with the current state of macroeconomics fundamentals
might be beneficial for the analysis of FX-rates modeling.


%Third Part - Motivation. Why is it important? Who cares?
% 
% Avoid overengineering and overcomplication. Provide simple setting
% Understand if macroeconomics variables are cointegrated at monthly lag frequency.
%
This methodological study stands therefore as a revision of the out of
sample failure of foreign exchange rates testing whether
macroeconomics variables are indeed incapable of explaining FX-rates
movements. This is achieved firstly by fitting the parametric models
to a comprehensive self collected data set enlarged in size compared
to the one used so far in the academic literature. Secondly, by
applying an innovative estimation technique that allows to explore the
more nuanced field of non-linear modeling by endogenously selecting
optimal structural breaks for fitting parametric models.

%Fourth Part - What we do in more details
Specifically, this study aims to explore the above at a monthly lag
frequency on a data set ranging from 1986 to 2006. This period allows
for a consistent estimation of all of the necessary parameters given
the chosen models. Moreover, it avoids modeling the most recent
financial crises where structural breaks are present in the series
according to the preliminary CUSUM test applied on the times series.\\
As in Meese and Rogoff (1983a and 1983b,
\cite{MeeseRogoffa,MeeseRogoffb}) I explore the out of sample
forecasting performance of foreign exchange rates for three major
monetary models. Firstly, the monetary models presented by Frenkel
(1976, \cite{Frenkel1976}) and Mussa\ (1976, \cite{Mussa1976})
claiming a link among monetary mass differentials, interest rates
differentials, output gap and FX-rates. Secondly the sticky price
model where prices adjust sluggishly to macroeconomics shocks,
exchange rates overshoot and inflation rates differential determine
the long run behaviour of FX-rates (Dornbush 1976,
\cite{Dornbusch}). Finally, a portfolio balance model claiming for the
importance of monetary flows and net current account differentials as
a key determinant for FX-rates changes as in Hooper and Morton (1982,
\cite{HooperMorton1982}).\\
Based on such monetary models six different stochastic models are
analyzed in order to forecast FX-rates. Three of them are presented in
the seminal work of Meese and Rogoff (1983a, \cite{MeeseRogoffa}) and
represent the benchmark to verify whether the results of the paper
hold for the time frame of interest. These models are a simple random
walk forecast, an OLS estimation without inclusion of lagged terms and
a vector auto-regressive model. On the top of these models I analyze
the performance of transfer models discussed in Montgomery and
Weatherby (1980, \cite{MontgomeryWeatherby}) and the vector error
correction model representation of the co-integration relationship
between macroeconomic variables and FX-rates proposed by Granger and
Engle (1987, \cite{EngleGranger}) as well as the previously mentioned
generalized tree structure model of Audrino and B{\"u}hlmann (2001,
\cite{AudrinoBuhlmann}).

%Fifth Part - Main Findings
If the results are in line with the seminal papers of Meese and Rogoff
in the case of univariate models, the picture resulting from a
multivariate fit contradicts the results of the reference papers. I do
not find enough evidence to support the claim of an underperformance
of multivariate linear models in comparison to a simple random walk
model. Moreover, I find a strong evidence for co-integration among
macroeconomic series and FX-rates for all of the analyzed country
series. This suggests a long term stable relation between the
fundamentals and the FX-rates rejecting the claim of Meese and Rogoff
about a weak relation between the two. Consistently with the last
claim, I do find that a vector error correction model representation
of the co-integration relation between macroeconomic series and
FX-rates outperforms the out-of-sample forecast of FX-rates at all
lags and for all of the country series. Finally, the applied
generalized tree structure model parametric evaluation statistically
outperforms all of the presented linear models for all country series
analyzed suggesting the importance of modeling non-linear terms in
modeling FX-rates and furthter contributing to the controversial debate of
the necessity of modeling non linear terms for the monetary modeling
of FX-rate series. An important breakthrough is moreover to notice
how both of the analyses results are consistent. Both underline
the special importance of modeling a long term stable
relation between macroeconomics monetary mass and foreign exchange
rates either trough a vector error correction model or a regime shift
strengthening the hypothesis that a link between monetary
models and FX-rates does indeed exist at monthly lag frequency given a
long enough sample period that allows for a reliable multivariate
endogenous model estimation as well the possibility to capture
important nonlinearities in the macroeconomics fundamentals influence
of FX-rates.

%Seventh part - This paper is organized as follows
The paper continues as follows. Section 2 outlines the methodological
approach and describes the techniques used for the empirical analysis
and the out-of-sample forecast. Section 3 continues by introducing the
data set used and the chosen proxies to capture the macroeconomic
fundamentals of interest, Section 4 reports the main results of the
empirical analysis and Section 5 concludes the findings.
