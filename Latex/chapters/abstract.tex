
%% This is the abstract of the Research Seminar.

\begin{abstract}
  This study reports an estimation of the out-of-sample performance of
  univariate, multivariate, and non-linear parametric monetary models
  contributing to the analysis of the Meese Rogoff paradox for the
  empirical CHF/USD, GBP/USD and JPY/USD series. 20 years of monthly
  observations and five models are analysed.  In line with the
  academic literature I find a random walk model to be the best
  univariate model to capture the unit root behaviour of FX-rates and
  to forecast the latter out-of-sample.  Nonetheless, in contrast with
  the general academic literature I do not find multivariate models
  to underperform a random walk model when forecasting FX-rates out of
  sample. Rather, I find that a vector error correction model
  representing the cointegration evidence among macroeconomics
  fundamentals and FX-rates marginally outperforms a random walk model
  for all the country series and estimation lags. Finally, the
  considereable out of sample performance of the generalized
  structured model points to non stable macroeconomic effects in time
  and non linear macroeconomic effects important to model in monetary
  models empirical studies.
\end{abstract}


% keywords can be removed
\keywords{FX-rates \and Monetary Models \and Meese- Rogoff Puzzle \and Cointegration \and Generalized Tree Structured Model}


%% do the coefficients shift in the different partitions??