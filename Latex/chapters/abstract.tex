
%% This is the abstract of the Research Seminar.

\begin{abstract}
  This study reports an estimation of the out-of-sample performance of univariate and
  multivariate monetary models contributing to the analysis of the Meese Rogoff paradox for
  the empirical CHF/USD, GBP/USD and YEN/USD series. 20 years of monthly observations and five
  models are analysed.
  In line with the academic literature we find a random walk model to be the best univariate model to
  capture the unit root behaviour of FX-rates and to forecast the latter out-of-sample.
  Nonetheless, in contrast with the general academic literature we do not find multivariate models to underperform
  a random walk model when forecasting FX-rates out of sample. Finally, we find that a vector error correction model
  representing the cointegration evidence among macroeconomics fundamentals and FX-rates outperforms a random walk
  model for all the country series and estimation lags.
\end{abstract}


% keywords can be removed
\keywords{FX-rates \and Monetary Models \and Meese- Rogoff Puzzle \and Cointegration \and VECM}
