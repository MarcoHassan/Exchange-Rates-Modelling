% Created 2019-05-17 ven 16:29
% Intended LaTeX compiler: pdflatex
\documentclass[presentation]{beamer}
\usepackage[utf8]{inputenc}
\usepackage[T1]{fontenc}
\usepackage{graphicx}
\usepackage{grffile}
\usepackage{longtable}
\usepackage{wrapfig}
\usepackage{rotating}
\usepackage[normalem]{ulem}
\usepackage{amsmath}
\usepackage{textcomp}
\usepackage{amssymb}
\usepackage{capt-of}
\usepackage{hyperref}
\usepackage{minted}
\usetheme{default}
\author{Marco Hassan}
\date{\today}
\title{Exchange-Rates-Modelling}
\hypersetup{
 pdfauthor={Marco Hassan},
 pdftitle={Exchange-Rates-Modelling},
 pdfkeywords={},
 pdfsubject={},
 pdfcreator={Emacs 26.1 (Org mode 9.1.13)}, 
 pdflang={English}}
\begin{document}

\maketitle
\begin{frame}{Outline}
\tableofcontents
\end{frame}

\alert{Work in progress..}

Download the data using quantmod. Multiple data provider: OECD, FED, SNB, BJP etc\ldots{}

\begin{frame}[label={sec:org1298dc9}]{Data cleaning:}
Step 1: detrend the data series.

Step 2: check the seasonality frequency using FFT

Step 3: remove seasonality

Step 4: Run unit root tests

Step 5: Perform structural break tests
\end{frame}

\begin{frame}[label={sec:org5f15c9b}]{Linear Modelling:}
Step 1: Random Walk forecasting with multiple lags.

Step 2: VAR 

2.1 Lag estimation via information criteria and likelihood ratio tests.

2.2 Model estimation via maximum likelihood. 

2.3 Variance decomposition of the implied impulse response and the within implied vector moving average model in order to check
for endogeneity in the series.

Step 3: ARIMAX
\end{frame}

\begin{frame}[label={sec:orgde0b548}]{Out of Sample Performance:}
\begin{block}{In detrended and seasonally adjusted series}
\latex



\begin{LaTeX}
\LaTeX
\end{LaTeX}
\end{block}
\end{frame}
\end{document}